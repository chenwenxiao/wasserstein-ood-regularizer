\documentclass[a3paper]{article}
\usepackage[hyphens]{url}
\usepackage[utf8]{inputenc}
\usepackage[pdftex]{graphicx}
\usepackage{epstopdf}
\usepackage{amsmath}
\usepackage{amssymb}
\usepackage{mathrsfs}
\usepackage{booktabs}
\usepackage[ruled,linesnumbered]{algorithm2e}
\usepackage{cleveref}
\usepackage[misc]{ifsym}
\usepackage{longtable}
\urlstyle{same}
\usepackage[a4paper, total={7in, 10in}]{geometry}
\usepackage{amsmath}
\usepackage{amssymb}
\usepackage{cleveref}
\usepackage{booktabs}
\usepackage[switch]{lineno} 
\usepackage{ntheorem}

\usepackage{float}
\restylefloat{table}

\newtheorem{theorem}{Theorem}
\newtheorem{lemma}{Lemma}
\newtheorem*{proof}{\it{Proof.}\rm}

\newcommand{\IE}{\textit{i.e.}, }
\newcommand{\EG}{\textit{e.g.}, }
\newcommand{\ET}{\textit{et al.}}
\newcommand{\ST}{\textit{s.t.}}

\newcommand{\dd}{\mathrm{d}}
\newcommand{\vv}[1]{\bm{\mathrm{{#1}}}}
\newcommand{\E}{\operatorname{\mathbb{E}}}
\newcommand{\EE}[1]{\operatorname{\mathbb{E}}\left[{#1}\right]}
\newcommand{\EEE}[2]{\operatorname{\mathbb{E}}_{{#1}}\left[{#2}\right]}
\newcommand{\KLD}{\operatorname{KL}}
\newcommand{\KLDD}[2]{\operatorname{KL}\left[{#1}\,\big\|\,{#2}\right]}
\newcommand{\abs}[1]{\left|#1\right|}
\newcommand{\Entropy}{\operatorname{H}}
\newcommand{\Entropyy}[1]{\operatorname{H}\left[#1\right]}
\newcommand{\pin}{p_{in}}
\newcommand{\pout}{p_{out}}
\newcommand{\pmix}{p_{mix}}
\newcommand*{\QED}{\hfill\ensuremath{\square}}  %自定义,空心
% Following comment is from ijcai97-submit.tex:
% The preparation of these files was supported by Schlumberger Palo Alto
% Research, AT\&T Bell Laboratories, and Morgan Kaufmann Publishers.
% Shirley Jowell, of Morgan Kaufmann Publishers, and Peter F.
% Patel-Schneider, of AT\&T Bell Laboratories collaborated on their
% preparation.

% These instructions can be modified and used in other conferences as long
% as credit to the authors and supporting agencies is retained, this notice
% is not changed, and further modification or reuse is not restricted.
% Neither Shirley Jowell nor Peter F. Patel-Schneider can be listed as
% contacts for providing assistance without their prior permission.

% To use for other conferences, change references to files and the
% conference appropriate and use other authors, contacts, publishers, and
% organizations.
% Also change the deadline and address for returning papers and the length and
% page charge instructions.
% Put where the files are available in the appropriate places.


\title{Technique Appendix}
\begin{document}

\pagestyle{plain} 

\section{Appendix A Counter-examples}

\begin{figure}[H]
\includegraphics[width=0.5\columnwidth]{figures/log_prob_histogram}
\includegraphics[width=0.5\columnwidth]{figures/log_prob_histogram-1}
\caption{Counter example of $\log p_\theta(x)$ with AUROC 0.0609 and 0.1104}
\end{figure}

\begin{figure}[H]
\includegraphics[width=0.5\columnwidth]{figures/log_prob_histogram-3}
\includegraphics[width=0.5\columnwidth]{figures/log_prob_histogram-4}
\caption{Counter example of $T_{perm}(x)$ with AUROC 0.3933 and 0.4692}
\end{figure}

\begin{figure}[H]
\includegraphics[width=0.5\columnwidth]{figures/grad_norm_histogram}
\includegraphics[width=0.5\columnwidth]{figures/grad_norm_histogram-1}
\caption{Counter example of $\|\nabla_x \log p_\theta(x)\|$ with AUROC 0.0196 and 0.0023}
\end{figure}

\begin{figure}[H]
\includegraphics[width=0.5\columnwidth]{figures/ll_with_complexity_histogram}
\includegraphics[width=0.5\columnwidth]{figures/ll_with_complexity_histogram-1}
\caption{Counter example of $S(x)$ with AUROC 0.2636 and 0.0062}
\end{figure}


\begin{figure}[H]
\includegraphics[width=0.5\columnwidth]{figures/kl_histogram}
\includegraphics[width=0.5\columnwidth]{figures/kl_histogram-1}
\caption{Counter example of $LLR(x)$ with AUROC 0.1054 and 0.5143}
\end{figure}

\begin{figure}[H]
\includegraphics[width=0.5\columnwidth]{figures/ll_waic_histogram}
\includegraphics[width=0.5\columnwidth]{figures/ll_waic_histogram-1}
\caption{Counter example of $WAIC(x)$ with AUROC 0.1041 and 0.2135}
\end{figure}

\begin{figure}[H]
\includegraphics[width=0.5\columnwidth]{figures/var_log_prob_histogram}
\includegraphics[width=0.5\columnwidth]{figures/var_log_prob_histogram-1}
\caption{Counter example of $\mathop{VAR}_\theta \log p_\theta(x)$ with AUROC 0.5358 and 0.3756}
\end{figure}

\begin{figure}[H]
\includegraphics[width=0.5\columnwidth]{figures/log_prob_histogram-5}
\includegraphics[width=0.5\columnwidth]{figures/log_prob_histogram-6}
\caption{Counter example of $\log p_\theta(x|y)$ with AUROC 0.3822 and 0.1150}
\end{figure}


\section{Appendix B Experiments}

The description of data is shown in the data appendix. 

The architecture of VAE is: 

\begin{table}[H]
\centering
\begin{tabular}{lllllll}
Layer     & Channel & Stride & Kernel & Activation  \\
\toprule
Resnet & 16 & 1x1 & 3x3 & Leaky ReLU \\
Resnet & 32 & 1x1 & 3x3 & Leaky ReLU \\
Resnet & 64 & 1x1 & 3x3 & Leaky ReLU \\
Resnet & 128 & 2x2 & 3x3 & Leaky ReLU \\
Resnet & 128 & 2x2 & 3x3 & Leaky ReLU \\
Resnet & 128 & 2x2 & 3x3 & Leaky ReLU \\
Flatten & \\
Dense & 256 &  &  & None \\
\bottomrule
\end{tabular}
\caption{Encoder architecture of VAE}
\end{table}

Before the decoder, we will use dense layer to map $z$ to a tensor with shape $8hw$ and reshape $(h / 4, w / 4, 128)$ where $h, w$ is the height and width of data. 

\begin{table}[H]
\centering
\begin{tabular}{lllllll}
Layer     & Channel & Stride & Kernel & Activation  \\
\toprule
Resnet Deconv & 128 & 2x2 & 3x3 & Leaky ReLU \\
Resnet Deconv & 128 & 2x2 & 3x3 & Leaky ReLU \\
Resnet Deconv & 128 & 2x2 & 3x3 & Leaky ReLU \\
Resnet Deconv & 64 & 1x1 & 3x3 & Leaky ReLU \\
Resnet Deconv & 32 & 1x1 & 3x3 & Leaky ReLU \\
Resnet Deconv & 16 & 1x1 & 3x3 & Leaky ReLU \\
Conv2d & 3 or 1 & 1x1 & 1x1 & None \\
\bottomrule
\end{tabular}
\caption{Decoder architecture of VAE}
\end{table}
We apply a Discretized Logistic Distribution as $p_\theta(x|z)$ and gaussian as $q_\phi(z|x)$. 


The architecture of PixelCNN is: 

\begin{table}[H]
\centering
\begin{tabular}{lllllll}
Layer     & Channel & Vertical Kernel & Horizontal Kernel& Activation & Dropout  \\
\toprule
Resnet & 64 & 2x3 & 2x2 & Leaky ReLU & 0.2\\
Resnet & 64 & 2x3 & 2x2 & Leaky ReLU & 0.2\\
Resnet & 64 & 2x3 & 2x2 & Leaky ReLU & 0.2\\
Resnet & 64 & 2x3 & 2x2 & Leaky ReLU & 0.2\\
Resnet & 64 & 2x3 & 2x2 & Leaky ReLU & 0.2\\
Resnet & 256 & 2x3 & 2x2 & Leaky ReLU & 0.2\\
Flatten & \\
Dense & 256 &  &  & None \\
\bottomrule
\end{tabular}
\caption{Architecture of PixelCNN}
\end{table}


The architecture of Wasserstein is: 

\begin{table}[H]
\centering
\begin{tabular}{lllllll}
Layer     & Channel & Stride & Kernel & Activation  \\
\toprule
Resnet Deconv & 128 & 2x2 & 3x3 & Leaky ReLU \\
Resnet Deconv & 128 & 2x2 & 3x3 & Leaky ReLU \\
Resnet Deconv & 128 & 1x1 & 3x3 & Leaky ReLU \\
Resnet Deconv & 64 & 1x1 & 3x3 & Leaky ReLU \\
Resnet Deconv & 32 & 1x1 & 3x3 & Leaky ReLU \\
Resnet Deconv & 16 & 1x1 & 3x3 & Leaky ReLU \\
Conv2d & 1 & 1x1 & 1x1 & None \\
\bottomrule
\end{tabular}
\caption{Generator architecture of WGAN}
\end{table}

\begin{table}[H]
\centering
\begin{tabular}{lllllll}
Layer     & Channel & Stride & Kernel & Activation  \\
\toprule
Resnet & 16 & 1x1 & 3x3 & Leaky ReLU \\
Resnet & 32 & 1x1 & 3x3 & Leaky ReLU \\
Resnet & 64 & 1x1 & 3x3 & Leaky ReLU \\
Resnet & 128 & 1x1 & 3x3 & Leaky ReLU \\
Resnet & 128 & 2x2 & 3x3 & Leaky ReLU \\
Resnet & 128 & 2x2 & 3x3 & Leaky ReLU \\
\bottomrule
\end{tabular}
\caption{Discriminator architecture of WGAN}
\end{table}

RealNVP has 3 level and 6 blocks at each level. Channel is 128 and activation is ReLU in each block. 

The hyper-parameters for VAE, PixelCNN, WGAN and Glow are shown in the scripts, shown in code appendix. 

The detailed experiments are shown in appendix D. 

The average runtime memory of VAE, PixelCNN, WGAN and Glow are 2.949GB, 2.851GB, 2.826GB and 2.917GB. 

\section{Appendix C Proofs}

We will derive the theorems in our papers by the assumptions:

\noindent \textbf{1.} The training data and testing data of in-distribution and out-of-distribution are i.i.d. .

\noindent \textbf{2.}  $div$ is a divergence , satisfying that $div(\pin, \pout) \gg 0$ and $div(\pout, \pin) \gg 0$.

\noindent \textbf{3.}  If $div(\pin, \pout)$ and $div(\pout, \pin)$ can be represented by sampling formula, \IE $div(\pin, \pout) \approx \sum_{i} f(x_i)$, then $f(x_i) \gg 0$ for almost $x_i$ in $\pin$.

\noindent \textbf{4.} $f: \mathcal{X} \rightarrow \mathcal{R}$ maps the distribution $\pin, \pout$ into two Gaussian distribution with constant variance.

\noindent \textbf{5.} Assumption 3 and 4 hold for any $\hat{f}$ approximating $f$.

\begin{lemma}\label{lemma1}
	Indicators $f_0$ and $f_1$ satisfying that if $f_0(x_1) < f_0(x_2)$ then $f_1(x_1) < f_1(x_2)$ and if $f_0(x_1) > f_0(x_2)$ then $f_1(x_1) > f_1(x_2)$ for $x_1, x_2$ in mixture distriution, have same performance for OoD. We call $f_0 \triangleq f_1$. 
\end{lemma}

\begin{proof}\rm
	Assume we get data $x_0, x_1, \ldots, x_n$ from mixture distribution. The metrics AUROC, AUPR, FPR@TPR95 and AP for indicator $f$ are all dependent on the order of $f(x_0), f(x_1), \ldots, f(x_n)$. The array $f_0(x_0), f_0(x_1), \ldots, f_0(x_n)$ and $f_1(x_0), f_1(x_1), \ldots, f_1(x_n)$ have the same order and then they have same performance for all metrics considered in OoD. \QED
\end{proof}

\begin{theorem}
	\label{thm1}
	$\log \pin(x) - \log \pout(x)$ is a symmetric indicator with great performance, \IE it reaches same performance in experiment A vs B and B vs A, with threshold zero.
\end{theorem}

\begin{proof}\rm
	We select $KL$ as $div$. By assumption 4, we have
	\begin{equation*}
		\log \pin(x_1) - \log \pout(x_1) \gg 0 \quad and \quad \log \pin(x_2) - \log \pout(x_2) \ll 0 
	\end{equation*}
	where $x_1 \sim \pin$ and $x_2 \sim \pout$. By assumption 5, above equation holds for $\log p_\theta(x) - \log p_\omega(x)$. Therefore, $\log \pin(x) - \log \pout(x)$ and $\log p_\theta(x) - \log p_\omega(x)$ can detect most of OoD, with threshold zero. 
	
	In experiment A vs B and B vs A, the indicator is $\log p_A(x) - \log p_B(x)$ and $\log p_B(x) - \log p_A(x)$. These two experiments have same mixture distribution. For any testing data $x_1, x_2$ in mixture distribution, if $\log p_A(x_1) - \log p_B(x_1) > \log p_A(x_2) - \log p_B(x_2)$, then $\log p_B(x_1) - \log p_A(x_1) < \log p_B(x_2) - \log p_A(x_2)$. Conversely, if $\log p_A(x_1) - \log p_B(x_1) < \log p_A(x_2) - \log p_B(x_2)$, then $\log p_B(x_1) - \log p_A(x_1) > \log p_B(x_2) - \log p_A(x_2)$. Thus, they have inverse order. 
	Noting that the positive and negative labels in experiment A vs B and B vs A are also inverse, $\log \pin(x) - \log \pout(x)$ reaches same performance in experiment A vs B and B vs A. \QED
\end{proof}

\begin{theorem}\label{thm2}
	For any mixture distribution $\pmix = \alpha \pin + \beta \pout$ where $\alpha + \beta = 1$ and $\alpha, \beta > 0$, the performance of indicator $\log \pin(x) - \log \pmix(x)$ and indicator $\log \pin(x) - \log \pout(x)$ is equal for OoD detection. 
\end{theorem}

\begin{proof}\rm
	For any $x1,x2$ satisfying $\log \pin(x_1) - \log \pout(x_1) < \log \pin(x_2) - \log \pout(x_2)$, we have $\log \pin(x_1) - \log \pmix(x_1) < \log \pin(x_2) - \log \pmix(x_2)$ by
	\begin{align*}
		&\log \pin(x) - \log \pmix(x) = -\log \frac{\alpha \pin(x) + \beta \pout(x)}{\pin(x)} = -\log 
		\Big(\alpha + \beta \frac{\pout(x)}{\pin(x)}\Big) \\
		&\log \frac{\pin(x_1)}{\pout(x_1)} < \log \frac{\pin(x_2)}{\pout(x_2)} \Rightarrow \frac{\pin(x_1)}{\pout(x_1)} < \frac{\pin(x_2)}{\pout(x_2)} \Rightarrow \frac{\pout(x_1)}{\pin(x_1)} > \frac{\pout(x_2)}{\pin(x_2)} \\
	 &\Rightarrow \log \Big(\alpha + \beta \frac{\pout(x_1)}{\pin(x_1)}\Big) > \log 
		\Big(\alpha + \beta \frac{\pout(x_2)}{\pin(x_2)}\Big) \\
		&\Rightarrow -(\log \pin(x_1) - \log \pmix(x_1)) > -(\log \pin(x_2) - \log \pmix(x_2)) \\
		&\Rightarrow \log \frac{\pin(x_1)}{\pmix(x_1)} < \log \frac{\pin(x_2)}{\pmix(x_2)} 
	\end{align*}
	
	By the same way, for any $x1,x2$ satisfying $\log \pin(x_1) - \log \pout(x_1) > \log \pin(x_2) - \log \pout(x_2)$, we have $\log \pin(x_1) - \log \pmix(x_1) > \log \pin(x_2) - \log \pmix(x_2)$.  By \cref{lemma1}, performance of indicator $\log \pin(x) - \log \pmix(x)$ and indicator $\log \pin(x) - \log \pout(x)$ is equal for OoD detection.
	
	\noindent \textbf{Remark.} In the proof, we do not use the condition that $\alpha > 0$ but only $\beta > 0$ and $\pmix(x) > 0$. It means that \cref{thm1} holds when $\alpha < 0, \beta > 0$ and $\pmix(x) > 0$. 
\QED
\end{proof}

\begin{theorem}\label{thm3}
When log-likelihood can detect OoD, \IE $\log \pin(x_1) > \log \pin(x_2)$ for almost $x_1 \sim \pin$ and $x_2 \sim \pout$, KL-based indicator can be also used to detect OoD.
\end{theorem}
\begin{proof}\rm
	Assume $\log \pout(x_1) < \log \pout(x_2)$, then $\log \pin(x_1) - \log \pout(x_1) > \log \pin(x_2) - \log \pout(x_2)$.
	KL-based indicator will get better performance than log-likelihood indicator. 
	
	Assume $\log \pout(x_1) > \log \pout(x_2)$, then likelihood indicator will make mistake in experiment B vs A (if current experiment is A vs B). Meanwhile KL-based indicator can detect OoD in both A vs B and B vs A by \cref{thm1}.  
	
	It means KL-based indicator is always better than log-likelihood indicator. 
\end{proof}

For the proofing of following theorems, we need an additional mathmatical simplification: by assumption 4 and 5, the indicators for OoD maps the distribution $\pin, \pout$ into two Gaussian distribution with constant variance, therefore, the AUROC between these two Gaussian distribution is dependent on the difference of their mean, \IE $\E_{\pin(x)} f(x) - \E_{\pout(x)} f(x)$ is our simplified metric for mathematical inference. 

\begin{theorem}\label{thm4}
For any likelihood-ratio indicator $\log \pin(x) - \log g(x)$ where $g$ is a continuous differentiable probability distribution, KL-based indicator outperforms them.
\end{theorem}
\begin{proof}\rm
	Consider the following optimization with subsidiary conditions for searching the optimal $g$ for likelihood ratio:
	\begin{align*}
	\max_g \Big\{\E_{\pin(x)} \big[\log \pin(x) - \log g(x)\big] &- \E_{\pout(x)} \big[\log \pin(x) - \log g(x)\big]\Big\} \\
	\textbf{s.t.} KL(\pin, g) = L \text{ and } &\int g(x) \dd x = 1 
	\end{align*}
	where $L$ is a constraint to the optimization domain of $g$. It could be solved by Lagrange multiplier method introduced by calculus of variation~\cite{gelfand2000calculus}. 
	Let $\lambda, \psi > 0$ be Lagrange Multiplier for the two constraint. The Lagrange function is 
	\begin{align*}
		J[g] &= \E_{\pin(x)} \big[\log \pin(x) - \log g(x)\big] - \E_{\pout(x)} \big[\log \pin(x) - \log g(x)\big] \\
		&-\lambda KL(\pin, g) - \psi \int g(x) \dd x \\
		&= \int \pin(x) \log \pin(x) - \pin(x) \log g(x) - \pout(x) \log \pin(x) \\ 
		&+ \pout(x) \log g(x) - \lambda \pin(x) \log \frac{\pin(x)}{g} - \psi g(x) \dd x = \int F(g, x) \dd x
	\end{align*}
	By calculus of variation, the extremal point for $J[g]$ satisfies that $\delta J = 0$, which is equal to
	\begin{align*}
		\frac{\partial F(g, x)}{\partial g} = 0 \Rightarrow -\frac{\pin(x)}{g(x)} + \frac{\pout(x)}{g(x)} + \frac{\lambda \pin(x)}{g(x)} - \psi = 0 \\
		\Rightarrow g(x) = \frac{1}{\psi} ((\lambda - 1)\pin(x) + \pout(x))
	\end{align*}
	Considering the subsidiary condition $\int g(x) \dd x = 1$, we have
	\begin{equation*}
		\int g(x) \dd x = \frac{1}{\psi} \int (\lambda - 1) \pin(x) + \pout(x) \dd x = \frac{1}{\psi} ((\lambda - 1) + 1) = \frac{\lambda}{\psi} = 1
	\end{equation*}
	Therefore, $\lambda = \psi$ and $g^*(x) = \frac{\lambda - 1}{\lambda} \pin(x) + \frac{1}{\lambda} \pout(x)$ is the extremal point.
	By the remark of \cref{thm1}, $\log \pin(x) - \log g^*(x)$ gets same performance as $\log \pin(x) - \log \pout(x)$ in OoD detection. 
	For any $L$, we can get $g^*(x)$, which has same performance as $\log \pin(x) - \log \pout(x)$. It means that KL-based indicator is the optimal indicator among all likelihood-ratio indicators. 
\end{proof}


\begin{theorem}\label{thm5}
	$\log \frac{p_\theta(x)}{p_\gamma(x)}$ can be used for detecting OoD. Moreover, $\log \frac{p_\theta(x)}{p_\gamma(x)}$ represents whether a sample $x$ in mixture distribution have been optimized in the training process of $\theta$.
\end{theorem}
\begin{proof}
	Let $p_{\hat{\gamma}}(x) = \alpha p_\theta(x) + \beta p_\omega(x)$. $\hat{\gamma}$ is the parameters that can be obtained by optimizer. 
	Considering the indicator $\log p_\theta(x) - \log p_{\hat{\gamma}}(x) = -\log (\alpha + \beta \frac{p_\omega(x)}{p_\theta(x)})$, for $x_1 \sim \pin$ and $x_2 \sim \pout$, we have 
	\begin{align*}
		\log \frac{p_\theta(x_1)}{p_{\hat{\gamma}}(x_1)} - \log \frac{p_\theta(x_2)}{p_{\hat{\gamma}}(x_2)} = \log (\alpha + \beta \frac{p_\omega(x_2)}{p_\theta(x_2)}) - \log (\alpha + \beta \frac{p_\omega(x_1)}{p_\theta(x_1)})
	\end{align*}
	We have known that $\log \frac{p_\theta(x_1)}{p_\omega(x_1)} \gg 0$ and $\log \frac{p_\theta(x_2)}{p_\omega(x_2)} \ll 0$ by \cref{thm1}. Then, we have $\log \frac{p_\theta(x_1)}{p_{\hat{\gamma}}(x_1)} - \log \frac{p_\theta(x_2)}{p_{\hat{\gamma}}(x_2)} > 0$. It means that $\log \frac{p_\theta(x)}{p_{\hat{\gamma}}(x)}$ can be used for detecting OoD.
	
	Next, consider the $\gamma$ is optimized better than $\hat{\gamma}$, \IE $\E_{\pmix} \log p_\gamma(x) \geq \E_{\pmix} \log p_{\hat{\gamma}}(x)$. 
	
	\begin{align*}
		\E_{\pmix} \log p_\gamma(x) = \alpha \E_{\pin} \log p_\gamma(x) + \beta \E_{\pout} \log p_\gamma(x)
	\end{align*}
	By $\E_{\pmix} \log p_\gamma(x) \geq \E_{\pmix} \log p_{\hat{\gamma}}(x)$, there are 3 case:
	
	\textbf{1.} $\E_{\pin} \log p_\gamma(x) \leq \E_{\pin} \log p_{\hat\gamma}(x)$ and $\E_{\pout} \log p_\gamma(x) \geq \E_{\pout} \log p_{\hat\gamma}(x)$. 
	
	\textbf{2.} $\E_{\pin} \log p_\gamma(x) \geq \E_{\pin} \log p_{\hat\gamma}(x)$ and $\E_{\pout} \log p_\gamma(x) \geq \E_{\pout} \log p_{\hat\gamma}(x)$. 
	
	
	\textbf{3.} $\E_{\pin} \log p_\gamma(x) \geq \E_{\pin} \log p_{\hat\gamma}(x)$ and $\E_{\pout} \log p_\gamma(x) \leq \E_{\pout} \log p_{\hat\gamma}(x)$. 
	
	By assumption 4, 5 and simplified metric, we know that if  $\E_{\pin} \log p_\gamma(x)$ increase, the KL-based indicator $\log \frac{p_\theta(x)}{p_{{\gamma}}(x)}$ will be worse and if  $\E_{\pout} \log p_\gamma(x)$ increase, the KL-based indicator $\log \frac{p_\theta(x)}{p_{{\gamma}}(x)}$ will be better. 
	
	Therefore, in case 1, the performance of $\log \frac{p_\theta(x_1)}{p_{\gamma}(x_1)}$ will be better than $\log \frac{p_\theta(x_1)}{p_{\hat\gamma}(x_1)}$. In case 2 and 3, we know $\E_{\pin} \log p_{\hat\gamma}(x) \leq \E_{\pin} \log p_\theta(x)$ and $\E_{\pin} \log p_\gamma(x) \leq \E_{\pin} \log p_\theta(x)$ since $\theta$ is well-trained for such loss function. 
	\begin{align*}
		\beta \E_{\pout} \log p_\gamma(x) = \E_{\pmix} \log p_\gamma(x) - \alpha \E_{\pin} \log p_\gamma(x) \\
		\geq \E_{\pmix} \log p_{\hat\gamma}(x) - \alpha \E_{\pin} \log p_\theta(x) \\
		= \alpha \E_{\pin} \log \frac{p_{\hat \gamma}(x)}{p_\theta(x)} + \beta \E_{\pout} \log p_{\hat \gamma}(x) \\
	\Rightarrow \E_{\pout} \log \frac{p_\gamma(x)}{p_{\hat\gamma}(x)} \geq \frac{\alpha}{\beta} \E_{\pin} \log \frac{p_{\hat \gamma}(x)}{p_\theta(x)}
	\end{align*}
	And we have 
	\begin{equation*}
		\E_{\pin} \log \frac{p_\gamma(x)}{p_{\hat\gamma}(x)} \leq \E_{\pin} \log \frac{p_\theta(x)}{p_{\hat\gamma}(x)}
	\end{equation*}
	Therefore, by our simplified metric, we have
	\begin{align*}
		(\E_{\pin(x)} \log \frac{p_\theta(x)}{p_{{\gamma}}(x)} - \E_{\pout(x)} \log \frac{p_\theta(x)}{p_{{\gamma}}(x)}) - (\E_{\pin(x)} \log \frac{p_\theta(x)}{p_{{\hat\gamma}}(x)} - \E_{\pout(x)} \log \frac{p_\theta(x)}{p_{\hat\gamma}(x)}) \\
		=  \E_{\pout} \log \frac{p_\gamma(x)}{p_{\hat\gamma}(x)} -\E_{\pin} \log \frac{p_{\gamma}(x)}{p_{\hat\gamma}(x)}\geq \frac{\alpha}{\beta} \E_{\pin} \log \frac{p_{\hat \gamma}(x)}{p_\theta(x)} - \E_{\pin} \log \frac{p_\theta(x)}{p_{\hat\gamma}(x)} \\
		= \frac{\alpha + \beta}{\beta} \E_{\pin} \log \frac{p_\theta(x)}{p_{\hat\gamma}(x)} = \frac{1}{\beta} \E_{\pin} \log \frac{p_\theta(x)}{p_{\hat\gamma}(x)} \geq 0
	\end{align*}
	It means that $\frac{p_\theta(x)}{p_{{\gamma}}(x)}$ is a better indicator than $\frac{p_\theta(x)}{p_{{\hat\gamma}}(x)}$ in cast 2 and 3. In conclusion,  $\frac{p_\theta(x)}{p_{{\gamma}}(x)}$ will be better and thus they can be used to detect OoD if $\gamma$ is trained better than $\hat{\gamma}$. 
	Additional, by the above proof, the most important property is that $\theta$ is well-trained in $\pin$ and thus $p_\gamma(x) < p_\theta(x)$ in $\pin$, \IE $\log \frac{p_\theta(x)}{p_\gamma(x)}$ represents whether a sample $x$ in mixture distribution have been optimized in the training process of $\theta$. 
\end{proof}

\begin{theorem}\label{thm6}
	Assumption 2 is a corollary of definition of OoD problem when $div$ is Wasserstein distance.
\end{theorem}
\begin{proof}
	By the definition of OoD, exist a $L > 0$, for any $x_1 \sim \pin, x_2 \sim \pout$, their distance $\|x_1 - x_2\|$ is at least $L$. 
	\begin{equation*}
		 W^1(\pin, \pout) = \min_{f \in \Gamma(\pin, \pout)} \iint f(x_1, x_2) \|x_1 - x_2\| \dd x_1 \dd x_2 \geq L
	\end{equation*}
\end{proof}

\begin{theorem}\label{thm7}
	$D(x)$ is a symmetric indicator.
\end{theorem}
\begin{proof}
	Let $D$ is the optimal discriminator in $W^1(\pin, \pout)$, for any $D'$ satisfying $Lip(D') \leq 1$, we have
	\begin{align*}
		\E_{\pout} D'(x) - \E_{\pin} D'(x) = \E_{\pin} [-D'(x)] - \E_{\pout} [-D'(x)] \\
		\leq  \E_{\pin} D(x) - \E_{\pout} D(x) =  \E_{\pout} [-D(x)] - \E_{\pin} [-D(x)] 
	\end{align*}
	Therefore, $-D(x)$ is the optimal discriminator in $W^1(\pout, \pin)$. By the same way in \cref{thm1}, 
	$D(x)$ is a symmetric indicator.
\end{proof}

\begin{theorem}\label{thm8}
	$\hat{D}$ that is optimal solution in $W^1(\pin, \pmix)$ is same to the optimal solution $D$ in $W^1(\pin, \pout)$. Moreover, the neural networks trained by $W^1(\pin, \pmix)$ and $W^1(\pin, \pout)$ will share the same optimization process.
\end{theorem}

\begin{proof}
	For any $D$ satisfying $Lip(D) \leq 1$, 
	\begin{align*}
		\E_{\pin} D(x) - \E_{\pmix} D(x) = \E_{\pin} D(x) - \alpha\E_{\pin} D(x) - \beta\E_{\pout} D(x) \\ 
		= (1 - \alpha)\E_{\pin} D(x) - \beta\E_{\pout} D(x)
		= \beta \E_{\pin} D(x) - \beta \E_{\pout} D(x) \\
		\leq \beta \E_{\pin} \hat{D}(x) - \beta \E_{\pout} \hat{D}(x) = (1 - \alpha)\E_{\pin} \hat{D}(x) - \beta\E_{\pout} \hat{D}(x) \\
		= \E_{\pin} \hat{D}(x) - \alpha\E_{\pin} \hat{D}(x) - \beta\E_{\pout} \hat{D}(x) = \E_{\pin} \hat{D}(x) - \E_{\pmix} \hat{D}(x)
	\end{align*}
	Therefore, $\hat{D}$ is also the optimal discriminator in $W^1(\pin, \pmix)$. Additional, in above proof, the loss functions for $D$ in $W^1(\pin, \pmix)$ and $W^1(\pin, \pout)$ are 
	\begin{equation*}
		\E_{\pin} D(x) - \E_{\pmix} D(x) = \beta (\E_{\pin} D(x) - \E_{\pout} D(x))
	\end{equation*}
	The two loss functions have same gradient direction for neural network optimization, thus the training optimization processes are same.
\end{proof}

\begin{theorem}\label{thm9}
	The discriminator in $W^1(\pin, \pout)$ is the best indicator among all indicators that is 1-Lipschitz. Moreover, it is the best indicator who has limited gradient.
\end{theorem}
\begin{proof}
	By our simplified metric, for any indicator $f$, 
	\begin{equation*}
		\E_{\pin} f(x) - \E_{\pout} f(x)
	\end{equation*}
	measures the performance of OoD detection. Thus 
	\begin{equation*}
		\max_{Lip(f) \leq 1} \E_{\pin} f(x) - \E_{\pout} f(x) = W^1(\pin, \pout)
	\end{equation*} 
	will search the best indicator among all indicators that is 1-Lipschitz. Especially, for a indicator $f$ has limited gradient $K$, function $g(x) = f(x) / K$ satisfies that $Lip(g) \leq 1$. By \cref{thm1}, $f \triangleq g$, and $g$ is worse than the optimal discriminator in $W^1(\pin, \pout)$. 
	
\end{proof}

\section{Appendix D Detailed Experiments}

\input{detailed_experiments}

%% The file named.bst is a bibliography style file for BibTeX 0.99c
\bibliographystyle{splncs04}
\bibliography{reference}


\end{document}

